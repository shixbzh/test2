\section{Objectifs du projet}
\label{ch:lit_rev} 

\subsection{Description plus précise de ce qu'il fallait faire} 

Dans un premier temps, il était question de développer une application informatique qui simule le jeu de la vie basé de Conway.
Pour cela il fallait un modèle du jeu simulable depuis le terminal, ensuite créer une représentation graphique de la grille et des cellules.

Pour ce faire, une étude et comprehension des règles du jeu de la vie était nécessaire.
En utilisant des concepts de programmation orientée objet pour traiter le comportement des membres du jeu. Aussi, devait être implementée la partie graphique conviviale permettant à l'utilisateur d'interagir avec la simulation.
Trouver une façon de gérer les états des cellules (vivant ou mort) selon les règles prédéfinies.
Réaliser des tests unitaires pour garantir le bon fonctionnement des classes et des méthodes.

Documentation et Rapport : Documenter le code source avec des commentaires et les contrats des differentes méthodes.
Rédiger un rapport détaillé expliquant la conception, l'implémentation, les résultats et les problèmes rencontrées

Présenter le projet de manière structurée avec des illustrations, des captures d'écran et des diagrammes explicatifs (diagrammes de classes, etc...)

\subsection{Description de travaux existants sur le même sujet}

Mise à part la publication originale du jeu par John Conway, plusieurs logiciels de simumlation du jeu de la vie ont vu le jour: 
Ces simulateurs permettent aux utilisateurs d'explorer les règles du jeu, d'observer les motifs émergents et de tester différentes configurations de cellules.
\begin{itemize}
    \item \href{https://neuralpatterns.io/}{Automates cellulaires neuraux}
    \item \href{https://www.geekpassion.fr/jeu-de-la-vie}{Jeu de la vie classique}
    \item \href{https://www.dcode.fr/jeu-de-la-vie}{Jeu de la Vie - Automate Cellulaire avec Règles} 

\end{itemize}

Aussi les Forums et Communautés en ligne de passionnés du "Jeu de la Vie". disposent de forums et sites web dédiésqui permettent aux utilisateurs de partager des créations, des astuces de programmation et des découvertes liées au jeu.

Comme indiqué précedemment en introduction, le jeu de la vie à plusieurs applications dans différents métiers:
Par exemple: 
\begin{itemize}
    \item En Physique : simulation d’écoulements, transfert de chaleur, résistance des matériaux, météorologie…
    \item En Chimie : optimisation géométrique de molécules complexes, mécanique moléculaire, dynamique moléculaire, diffusion de molécules dans des solides poreux, simulation de polymères, transitions de phase…
    \item En Biologie : modélisation de grosses molécules (ADN, protéines, etc), de leur configuration spatiale, de leurs interactions, simulation de comportements collectifs (fourmis, bactéries, etc)…
    \item En Informatique : réseaux de neurones, automates cellulaires, vie et intelligence artificielles…
    \item En Mathématiques : logique, indécidabilité, phénomènes chaotiques…
\end{itemize}







