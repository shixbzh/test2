\section{Conclusions}
\label{ch:con}
\subsection{Récapitulatif des fonctionnalités principales}
Pour résumer, comme vous l'aurez remarqué, le jeu de la vie n'est pas un jeu à proprement dit: notre projet combine les concepts du jeu de la vie de Conway avec une interface graphique Java(swing) pour offrir une expérience interactive, digne des meilleurs simulateurs du jeu de la vie sur le net. 
Les fonctionnalités majeures incluent la simulation automatisée des règles du jeu, la possibilité d'interaction directe avec la grille et une interface utilisateur simple mais efficace pour contrôler la simulation, sans oublier que le jeu est directement jouable depuis le terminal.

L'objectif principal était de fournir une implémentation visuellement attrayante du jeu de la vie, donnant possibilité à l'utilisateur de découvrir les principes de l'automate cellulaire de Conway.


\subsection{Propositions d’améliorations}
Déjà, ce qui à été fait: 
La grille affiche un état initial aléatoire, chargé ou vide.
Les règles de Conway sont appliquées pour faire évoluer les cellules.
L'utilisateur peut démarrer, mettre en pause et réinitialiser la simulation.

Interaction Utilisateur :
L'utilisateur peut cliquer sur les cellules pour les faire naître ou mourir.
Les cellules vivantes sont représentées en blanc et les cellules mortes en noir.
Interface Graphique :
L'interface utilise Swing pour créer une fenêtre graphique conviviale.
Des boutons permettent de contrôler la simulation (démarrer, mettre en pause, réinitialiser).
Mise à Jour Dynamique :
La vue de la grille est mise à jour dynamiquement à chaque étape de simulation.
Les changements d'état des cellules sont reflétés graphiquement en temps réel.

    Pour améliorer le simulateur du jeux de la vie, quelques améliorations peuvent être apportés:
    \begin{itemize}
        \item Permettre à l'tilisateur de choisir la taille de sa grille
        \item Permettre à l'utilisateur de choisir les couleurs des cellules mortes ou vivantes
        \item Choisir la règle depuis une liste déroulante
        \item Trouver une solution pour la bar de scroll
        \item Faire fonctionner le HashLife
    \end{itemize}